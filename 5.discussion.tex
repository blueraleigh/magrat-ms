We described a method for estimating locations of genetic ancestors from 
genome-wide genealogies. The method is based on minimizing the overall number 
of migration events (or overall migration distance) necessary to explain the 
geographic distribution of sampled genomes, optionally including information 
from genealogy branch lengths in the optimization. Simulations indicate that 
the method achieves reasonable accuracy over short and moderate timescales, and 
an analysis of contemporary human genomes demonstrates that the method recovers 
broad patterns in the history of human migration out of Africa and across Asia 
and Europe.

\subsection{Relation to existing work}

Several existing approaches to geographic inference with tree sequences merit 
discussion in relation to our own work. \citet{Wohns_etal_2022} introduced a
nonparametric approach that estimates ancestor locations by successively
averaging the coordinates of sample locations in a postorder traversal of the
ARG to their most recent common ancestor. The resulting estimates are local 
estimates in the sense that the inferred location of an ancestor depends only 
on the locations of samples that trace some portion of their ancestry to that 
ancestor and on the topology of the corresponding subset of the ARG. By contrast,
our approach estimates the location of an ancestor using information from all
samples. Because all samples share common ancestry at some time in the past, 
even those samples that are not direct descendants of an ancestor can be 
informative about that ancestor's location.

\citet{Osmond_Coop_2021} describe a likelihood method for locating genetic
ancestors and estimating migration rates that is based on a model of branching
Brownian motion. Their approach also uses information from all samples to
estimate ancestral locations and can optionally estimate separate migration
rates for deep and shallow time horizons. Unlike our method, inference is 
carried out on a sample of widely spaced genealogies rather than on the full 
tree sequence. We note that when squared Euclidean distance weighted by inverse 
branch length is used as the transition cost function, the maximum parsimony 
reconstruction on each marginal genealogy has highest posterior probability 
under a Brownian motion dispersal process. In this sense, the genome-wide 
average reconstruction produced by our method can be viewed as a weighted 
average of posterior modes.

% maybe a section here about branch lengths: to use or not to use?

% another section about distinguishing between ongoing migration vs historical
% association?

\subsection{Limitations}

We note that the approach we have described assumes that lineages disperse
independently of one another and that the dispersal process is independent of
the coalescent process between coalescent events in a reconstructed gene tree. 
That is, a reconstruction depends only the transition cost function and on the 
topology (and possibly branch lengths) of a genealogy. Moreover, the parsimony 
algorithm can enumerate only a small subset of migration histories that may 
actually have occurred. For these reasons, we expect migration rate estimates 
based on parsimony statistics to be biased, especially in scenarios where 
migration is high. Finally, all methods of inference will depend on our
ability to accurately estimate the marginal genealogies in the tree
sequence. To the extent that errors in gene tree inference are random with
respect to the geographic distribution of the samples, they should lead to
higher migration rate estimates and increased uncertainty in the geographic
placement of ancestors.



