\subsection{Main approach: computing minimum migration fits to local trees}

Given a set of sampled genomes with known locations and a tree sequence that
relates the samples, our approach efficiently reconstructs the locations of 
genetic ancestors using a minimum migration heuristic. Conceptually, our
method may be thought of as proceeding in three steps
(Fig. \ref{fig:concept-fig}).

In the first step, a minimum migration cost function $f_{uk}$ is fitted to each
node $u$ in each local tree $k$ in which $u$ appears using the generalized
parsimony algorithm \citep{Sankoff_1975, Sankoff_Rousseau_1975}. For each 
spatial location, $f_{uk}(x)$ then returns the smallest sum of migration costs
between all ancestor-descendant pairs in local tree $k$ that can explain the
sampled locations when $u$ is at location $x$. We assume migration cost is a function
of migration distance and provide three cost function implementations. In each 
of these three cases we allow migration distances to optionally be weighted by 
inverse branch length. For a continuous state space, we use either the squared
Euclidean distance or the Manhattan distance. Squared Euclidean distance leads
to cost functions that are convex quadratic functions of the spatial locations
\citep{Maddison_1991}; Manhattan distance, to convex piecewise linear functions
\citep{Csuros_2008}. For a finite set of geographic locations, a square matrix 
with arbitrary transition costs between all pairs of geographic states leads to
minimum migration costs that are discrete functions defined on the set of
geographic locations \citep{Clemente_etal_2009}.

In the second step, we form a new function $F_u$ by taking a weighted average 
of $f_{uk}$ over the local trees in which $u$ appears, using the genomic spans
of each local tree as weights. We interpret $F_u$ as the minimum migration cost
of an average ancestral base pair. In other words, if one were to take each
sampled base pair that inherits from $u$ and fit a minimum migration cost function to its
coalescent history, the average of those would look like $F_u$. In the Appendix
we present a detailed description of the algorithm we use for computing
$F_u$ from the local tree fits $f_{uk}$. Considerable computational savings
can be achieved by using the succinct tree sequence encoding
\citep{Kelleher_etal_2016}, which allows us to efficiently maintain
the state of parsimony calculations as we iterate over local trees. As detailed
in the Appendix, the local tree fits also allow us to compute an estimate of
effective migration rate, which we define as the mean lineage migration cost 
of an average base pair. That is, if one were to take the coalescent history of
each sampled base pair and compute the mean the per-branch migration cost in a
most parsimonious migration history, the average of those -- taken over all
sampled base pairs -- would equal the effective migration rate.

Finally, we locate each ancestor by finding the spatial location $x$ where
$F_u(x)$ is smallest. A constrained (rather than global) minimization of $F_u$
could also be used to locate ancestors if some spatial locations are known to
be uninhabitable. Although we do not pursue this possibility in the current
study, uncertainty in ancestor locations could be explored by sampling from
$\exp\bigl(-\lambda F_u(x)\bigr)$, where $\lambda$ is a tuning parameter that
controls how strongly deviations from the spatial location that minimizes the
overall migration cost are penalized.

\subsection{Spatiotemporal ancestry coefficients}

We use our ancestor location estimates to define two spatially and
temporally explicit ancestry coefficients as follows. In the first case,
consider a subset $i$ of the sampled genomes. Define $A_i(t)$ to be the total
amount of sample material in the subset that is descended from ancestral
material at a point in time $t$ in the past. Define $A_{ik}(t)$ to be the total
amount of sample material descended from ancestral material in geographic
state $k$ at time $t$ in the past. We then define the spatiotemporal ancestry
coefficient for subset $i$ as $q_{ik}(t) = \frac{A_{ik}(t)}{A_i(t)}$, the fraction of
sample material that is descended from ancestral material in geographic
state $k$ at time $t$ in the past. Thus, if we were to pool all sample material
in subset $i$ whose coalescent history extended at least $t$ time units into the
past, $q_{ik}(t)$ is the probabily that we can trace the coalescent history of a
randomly selected base pair from that pool to an ancestral base pair in
geographic state $k$ at time $t$ in the past.

Next, define $A_i(t_l,t_r)$ to be the total amount of sample material in the
subset that is descended from ancestral material present during an interval in
time $[t_l,t_r)$ in the past. Define $A_{ijk}(t_l,t_r)$ to be the total amount
of sample material descended from ancestral material that migrated from
geographic state $j$ to state $k$ during an interval in time $[t_l, t_r)$ in
the past (where $j$ is not equal to $k$). We then define the spatiotemporal
ancestry flux coefficient for subset $i$ as 
$q_{ijk}(t_l, t_r) = \frac{A_{ijk}(t_l, t_r)}{A_i(t_l, t_r)}$. Thus, if we were
to pool all sample material in subset $i$ whose coalescent history extended at
least $[t_l,t_r)$ time units into the past, $q_{ijk}(t_l,t_r)$ is the probabily
that we can trace the coalescent history of a randomly selected base pair from
that pool to an ancestral base pair that migrated from geographic state $j$ to
state $k$ during the interval $[t_l,t_r)$ in the past.

As defined, our spatiotemporal ancestry coefficients require knowing the geographic
locations of ancestral lineages at arbitrary points in time in the past but our
minimum migration location estimates (i.e., $\arg\min_x F_u(x)$) are available
only for the two endpoints of an ancestral lineage. By using the migration cost
function and conditioning on these endpoint states, however, we can sample a
minimum cost migration history for each lineage and locate ancestors at
arbitrary points in time. Further details about this procedure are presented in
the appendix.


\subsection{Simulation study: testing ancestor location accuracy}

We tested performance of the method using spatially explicit forward-time 
simulations in SLiM v4.0.1 \citep{Haller_Messer_2023}. Simulated individuals
were semelparous and hermaphroditic and diploid for a single chromosome with 
$10^8$ basepairs and a recombination rate of $10^{-8}$ per basepair per
generation. Individuals coexisted on a two-dimensional square plane with
reflecting boundaries and a side length of $1000\sigma$ units, where $\sigma$ 
was the standard deviation of the dispersal kernel.

Population regulation occurred via density dependent effects on fecundity and
survival. The number of offspring produced by a focal individual in generation
$t$ was Poisson-distributed with a mean equal to 
$\frac{R}{1 + \frac{N_t \times (R-1)}{K}}$,
where $N_t$ is the local population density in a circle of radius $3\sigma$ 
around the focal individual at generation $t$, $R$ is the population growth 
rate at low density, and $K$ is the local carrying capacity density. During 
reproduction, each individual (the ``mother") chose a mate (the ``father") 
uniformly at random from the set of individuals living within a radius of 
$3\sigma$ from itself. A random location centered on the mother's location was 
then chosen for each offspring by drawing from a dispersal kernel with a 
standard deviation of $\sigma$. Each offspring produced in generation $t$ 
survived to reproduce in generation $t+1$ with probability 
$\min(1.0, \frac{K}{O_t})$, where $O_t$ is the local population density in a 
circle of radius $3\sigma$ around the focal offspring in generation $t$.

We conducted simulations using either a Gaussian dispersal kernel or a double
exponential dispersal kernel and 10 different $\sigma$ levels that were equally
spaced from 0.2 to 2.0. The Gaussian kernel emulates a Brownian motion while
the double exponential kernel emulates a Laplace motion, which generates more
extreme displacements of offspring relative to Brownian motion.

We conducted 10 replicate simulations for each $\sigma$ level under each 
dispersal kernel. For all simulations, we set $R=2$ and $K=\frac{30}{(3\sigma)^2\pi}$.
Each simulation was initiated from a randomly located founder population of 30
individuals that was allowed to grow and disperse for 10000 generations, 
after which the simulation was terminated and the recorded (simplified) tree
sequence of extant individuals was saved for analysis.

Simulations conducted under the Gaussian dispersal kernel were analyzed using
squared Euclidean distance weighted by inverse branch length as the transition
cost function. Simulations conducted under the double exponential dispersal
kernel were analyzed using Manhattan distance weighted by inverse branch length
as the transition cost function. For all simulations we evaluated our ability
to accurately estimate effective migration rates and the locations of 
genetic ancestors. We defined the true effective migration rate ($\sigma_e$) to 
be the maximum likelihood estimate of $\sigma$ under each dispersal kernel
assuming we had complete knowledge of the locations of all ancestral individuals
in the tree sequence.

\subsection{Empirical example: reconstructing human migration out of Africa}

We applied our method to a contemporary sample of 1070 georeferenced human
genomes from the Human Genome Diversity Project using the dated tree sequences
inferred by \citet{Wohns_etal_2022}. Geographic coordinates of sampled genomes
were first converted to Cartesian coordinates on the surface of an authalic
sphere, after which we computed $F_u$ for each node in the empirical tree sequence
using squared Euclidean distance weighted by inverse branch length as the
migration cost function. For each point in time we can reconstruct the locations
of genetic ancestors as follows. Let $E(t)$ denote the set of lineages existing
at time $t$ and let $A$ denote a set of locations those lineages can potentially
inhabit. Then, for each edge $e = (u, v)$ in $E(t)$ we form 
$F_e = (1 - \frac{t_u - t}{t_u - t_v})F_u + \frac{t_u - t}{t_u - t_v}F_v$
by linear interpolation between ancestor $u$ and descendant $v$ and locate
the lineage at time $t$ as $\arg\min_{a \in A} F_e(a)$.

Using this procedure we created a set of spatial-temporal bins to explore
historical geographic patterns in human genetic ancestry over the last 
one-half million years. Equal-interval time bins spanning 100 generations 
(or 2500 years using a 25 year generation time) extended from the present
to 20000 generations in the past. An equal area discrete global grid 
\citep{Barnes_Sahr_2023} (cell spacing approximately 450 km) intersected with
Earth's landmass provided a set of habitable locations which were subsequently
mapped to 15 broad geographic regions for the purposes of ancestry
assignment.




