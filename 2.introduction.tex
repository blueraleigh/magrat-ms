The genomes of a sample of present-day individuals existed at every point in the past, 
scattered across geographic space and contained within the ancestors 
from whom they will eventually inherit their genetic material.
The way those ancestors moved across space through history 
determines spatial patterns of genetic relatedness in the present \citep{Bradburd_Ralph_2019}.
Understanding these spatial patterns is vital both for 
the identification of the genomic basis of phenotypic variation 
and for 
knowledge of the demographic history of a species.
In humans, 
complex, spatial patterns of genetic variation and relatedness 
are often summarized with crude, discrete labels, 
such as ``genetic ancestry groups" \citep{Coop_2022}, 
which, although loosely based on geographic history, 
sacrifice accuracy in the name of expedience and utility.
However, if we knew the locations of the ancestors of a sample, 
we could much more precisely report the geographic ancestry 
of a set of modern-day samples through time.
Moreover, we could learn about the history of dispersal, 
identifying major population movements, 
demographic events, and 
barriers to migration.

Here, we present a method for efficiently computing 
minimum migration statistics over genome-wide genealogies, 
allowing us to infer the geographic locations of shared ancestors 
of a modern, georeferenced sample.
Our method 

%
%The ancestry of a sample of recombining genomes can be described by a complex
%network recording the history and timing of coalescent and recombination
%events. This network -- an ancestral recombination graph (ARG) -- can be
%factored into a sequence of genealogical trees that record the coalescent 
%history of the sample for the different non-recombinant genomic regions. The
%tree sequence contains a wealth of population genetic information about the sample 
%\citep{Ralph_etal_2020}, and recent advancements in the inference and storage 
%of tree sequences have made them a powerful tool for learning about population 
%processes and historical events shaping patterns of genetic variation
%\citep{Anderson-Trocme_etal_2023, Kelleher_etal_2016, Kelleher_etal_2019,
%Lewanski_etal_2023, Wong_etal_2023}.
%
%The gene trees (often called local trees) that comprise a tree sequence are
%embedded within a pedigree formed by the dispersal and reproduction of
%individuals in a landscape. If we knew the pedigree and locations of ancestral 
%individuals it would be a simple matter to determine how movement patterns have
%varied over space and time. Unfortunately, the pedigree and its geographic
%context are, in most cases, unknown to us. However, the pedigree is
%intrinsically shaped by the geography of individuals \citep{Bradburd_Ralph_2019},
%so from the geographic distribution of sampled genomes on a tree sequence we
%can learn about the locations and movements of genetic ancestors at different
%times in the past. The intuition is that most dispersal and reproduction is
%geographically constrained: nearby individuals tend to be more closely related
%than far away individuals \citep{Wright_1943}.  The spatial locations of sampled
%genomes, together with the genealogies that relate them, are therefore expected
%to be informative about the locations of ancestral genomes
%\citep{Wakeley_1999, Wilkins_2004}.
%
%For a non-recombining sequence, simple statistics on the minimum number of
%migration events consistent with the gene tree and sample locations have
%proven useful for estimating migration rates and other important population
%genetic quantities, such as isolation by distance and Wright's neighborhood
%size \citep{Slatkin_Maddison_1989, Slatkin_Maddison_1990}. For a recombining
%sequence this approach is complicated by the fact that the gene tree on which
%the minimum migration statistic is computed necessarily represents some sort of
%average of the gene trees of the nonrecombined segments. The work of 
%\citet{Hudson_etal_1992} has shown that migration rate estimates based on
%minimum migration statistics are surprisingly robust in the presence of
%recombination -- often outperforming estimates based on $F_{ST}$ -- but 
%suggested a better approach would be to average minimum migration 
%statistics across gene trees estimated for each nonrecombined segment.
%
%Our main contribution in this paper is to show how we can use recent
%advancements in the storage of tree sequences to efficiently compute and
%average minimum migration statistics over genome-wide genealogies. Specifically,
%we demonstrate how the generalized parsimony algorithm
%\citep{Sankoff_1975, Sankoff_Rousseau_1975} can be used with the succinct tree 
%sequence encoding \citep{Kelleher_etal_2016} to estimate migration rates and 
%locations of genetic ancestors in continuous and discrete geographic space.
%We test this method using spatially explicit forward-time simulations. We then 
%apply this method to a continental dataset of humans. Our work adds to a
%growing body of research using tree sequences for inference in spatial 
%population genomics.





