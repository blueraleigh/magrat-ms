\subsection*{A1. Overview}

We first review generalized parsimony as it applies to a single gene tree
\citep{Sankoff_1975, Sankoff_Rousseau_1975}. For each node $u$ we maintain
three cost functions. The \textit{node cost} function $g_u(x)$ assigns a cost
to each geographic state $x$ (when $u$ is in state $x$) that gives the minimum
migration cost required to explain the geographic states of all sample nodes
whose most recent common ancestor is $u$. If $u$ is a sample node, we set
$g_u(x) = 0$ if $x = x_u$ and $g_u(x) = \infty$ if $x \neq x_u$. When $u$ is 
not a sample node, $g_u$ is formed by the sum of the \textit{stem cost}
functions of its children.

The stem cost function for a node $u$ is given by
%
$h_u(x) = \min_z \bigl[ \Delta_u(x, z) + g_u(z) \bigr]$.
%
The function $\Delta_u(x, z)$ assigns a cost to the migration from state $x$ to
state $z$ over the branch leading to node $u$. Here, we assume that $\Delta_u$
can be factored as:
%
$\Delta_u(x,z) = \phi(\tau_u) * \Delta(x,z)$,
%
where $\tau_u$ is the length of the branch leading to node $u$ and $\Delta$ is
a time-independent transition cost function. Typical choices for the
time-dependent component are $\phi(\tau_u) = \frac{1}{\tau_u}$ or 
$\phi(\tau_u) = 1$, the latter making the whole cost function independent of
branch lengths.

The node cost and stem cost functions can be computed in a single post-order
tree traversal. A second pre-order tree traversal can then be used to compute
the \textit{final cost} functions for each node. The final cost $f_u(x)$
gives the minimum migration cost required to explain the geographic states of
\emph{all} sample nodes when $u$ is in state $x$. When $u$ is the root of the
tree, $f_u(x) = g_u(x)$. Otherwise, 
%
$f_u(x) = \min_z \bigl[ f_{\pi_u}(z) - h_u(z) + \Delta_u(z,x) \bigr] + g_u(x)$,
%
where $\pi_u$ represents the immediate ancestor of node $u$.

\iffalse
To sample a most parsimonious dispersal history, we first choose a state at the
root of the tree that minimizes the final cost function. We then recurse through 
the tree in a pre-order traversal and, for each node $u$, we choose a state
$x_u = \arg\min_x \bigl[\Delta_u(x_{\pi_u}, x) + g_u(x)\bigr]$. Depending on the
nature of the geographic state space and cost function there may be multiple
(sometimes many) most parsimonious dispersal histories. We note that the 
minimum cost ancestor location assignments produced by a parsimony 
reconstruction can be viewed as a maximum posterior probability reconstruction 
when the cost function $\Delta_u$ is proportional to the negative logarithm of 
the lineage transition probability of a stochastic dispersal process 
\citep{Maddison_1991}.

Alternatively, if we are not interested in only the most parsimonious histories 
we could instead choose a state for node $u$ with probability 
$\mathbb{P}(X_u = x | X_{\pi_u} = x_{\pi_u}) \propto \exp{\bigl[-\Delta_u(x_{\pi_u}, x) - g_u(x)\bigr]}$.
\fi

\subsection*{A2. Algorithm}

Here we describe how we use the succinct tree sequence encoding
\citep{Kelleher_etal_2016} to efficiently compute node cost and stem cost 
functions for each local genealogy. Under the coalescent with recombination, 
nearby gene trees are highly correlated. Moving between adjacent local trees 
typically requires only a small number of subtree-prune-and-regraft (SPR) 
operations. By keeping an index of the edges involved in these SPR operations 
and the order in which they need to be applied, the succinct tree sequence 
encoding allows us to efficiently maintain the state of parsimony calculations 
as we iterate over the local genealogies.

A detailed description of the tree sequence data structure can be found in
\citet{Kelleher_etal_2016}. For our purposes, it suffices to know that
relationships among all $N$ nodes in the tree sequence are recorded in the edge 
table $E$. Each row in the edge table records an ancestor-descendant
relationship, and the indices of nodes involved in that relationship can be 
accessed as $E[k].\texttt{parent}$ and  $E[k].\texttt{child}$. The relationship 
encoded by an edge applies to the half-open genomic interval between $E[k].\texttt{left}$ 
(inclusive) and $E[k].\texttt{right}$ (exclusive). As we move along the genome 
from left to right, the index vectors $I$ and $O$ give the insertion and
removal order of the edges needed to build each marginal gene tree topology 
(recorded in the vector $\pi$). The following algorithm is based on
the branch statistic algorithm that appears in \citet{Ralph_etal_2020} but
differs in its treatment of sample weights and how they are propagated along
a genealogy.

\subsubsection*{Algorithm P (generalized parsimony)}

Given a set of georeferenced samples related by a tree sequence with length
$L$, compute the genome-wide average final cost function $F_u(x)$ and the
genome-wide average migration rate statistic $\sigma_P$. The function $F_u(x)$ 
returns the average minimum migration cost required to explain the observed 
geographic states of all sample nodes when $u$ is in state $x$, where the 
average is taken over all local genealogies (weighted by their genomic span)
where node $u$ appears. The statistic $\sigma_P$ is the average per-branch 
migration cost in a most parsimonious migration history averaged over all
local genealogies weighted by their genomic span.

\begin{description}

\item[S1.] [Initialization.]
    For $0 \leq u < N$
        set $\pi_u \leftarrow -1$,
        $s_u \leftarrow 0$,
        $g_u(x) \leftarrow 0$,
        $F_u(x) \leftarrow 0$.
        Then,
        if $u$ is a sample node and $x \neq x_u$,
            set $g_u(x) \leftarrow \infty$ and $F_u(x) \leftarrow \infty$

    Finally,
        set $j \leftarrow 0$,
        $k \leftarrow 0$,
        $\sigma_P \leftarrow 0$,
        $t_l \leftarrow 0$,
        $s_P \leftarrow 0$.

\item[S2.] [Terminate.]
    If $j = |E|$ terminate.

\item[S3.] [Edge removal loop.]
    If $k = |E|$ or $t_l \neq E[O_k].\texttt{right}$ go to S6.

\item[S4.] [Remove edge.]
    Set $u \leftarrow E[O_k].\texttt{parent}$,
        $v \leftarrow E[O_k].\texttt{child}$,
        $w \leftarrow \pi_u$, and
        $k \leftarrow k + 1$.
        Then,
        if $w \neq -1$,
            set $g_w(x) \leftarrow g_w(x) - h_u(x)$.
        Finally,
        set $g_u(x) \leftarrow g_u(x) - h_v(x)$,
        $\pi_v = -1$,
        $v \leftarrow u$,
        $u \leftarrow w$.

\item[S5.] [Update node and stem costs.]
    While $u \neq -1$,
        set $w \leftarrow \pi_u$ and
        if $w \neq -1$
            set $g_w(x) \leftarrow g_w(x) - h_u(x)$.
        Then set
        $h_v(x) \leftarrow \min_{z}\bigl[\Delta_v(x,z) + g_v(z)\bigr]$,
        $g_u(x) \leftarrow g_u(x) + h_v(x)$,
        $v \leftarrow u$, $u \leftarrow w$.
    Afterward, go to S3.

\item[S6.] [Edge insertion loop.]
    If $j = |E|$ or $t_l \neq E[I_j].\texttt{left}$ go to S9.

\item[S7.] [Insert edge.]
    Set $u \leftarrow E[I_j].\texttt{parent}$,
        $v \leftarrow E[I_j].\texttt{child}$,
        $w \leftarrow \pi_u$, and
        $j \leftarrow j + 1$.
        Then,
        if $w \neq -1$,
            set $g_w(x) \leftarrow g_w(x) - h_u(x)$.
        Then set
        $h_v(x) \leftarrow \min_{z}\bigl[\Delta_v(x,z) + g_v(z)\bigr]$.
        Finally,
        set $g_u(x) \leftarrow g_u(x) + h_v(x)$,
        $\pi_v = u$,
        $v \leftarrow u$,
        $u \leftarrow w$.

\item[S8.] [Update node and stem costs.]
    While $u \neq -1$,
        set $w \leftarrow \pi_u$ and
        if $w \neq -1$
            set $g_w(x) \leftarrow g_w(x) - h_u(x)$.
        Then
        set $h_v(x) \leftarrow \min_{z}\bigl[\Delta_v(x,z) + g_v(z)\bigr]$,
        $g_u(x) \leftarrow g_u(x) + h_v(x)$,
        $v \leftarrow u$, $u \leftarrow w$.
    Afterward, go to S6.

\item[S9.] [Genomic span of tree.]
    Set $t_r \leftarrow L$.
    If $j < |E|$ set
        $t_r \leftarrow \min\bigl(t_r, E[I_j].\texttt{left}\bigr)$.
    Then,
    if $k < |E|$ set
        $t_r \leftarrow \min\bigl(t_r, E[O_k].\texttt{right}\bigr)$.
    Set $s \leftarrow t_r - t_l$.

\item[S10.] [Update average migration costs.]
    Set $\sigma \leftarrow 0$, $n \leftarrow 0$, $s_P \leftarrow s + s_P$.
    Then visit each node $u$ in a pre-order traversal and set
    $s_u \leftarrow s_u + s$.
    If $\pi_u = -1$ set
        $f_u(x) \leftarrow g_u(x)$ and set
        $\sigma \leftarrow \sigma + \min_z f_u(z)$;
    otherwise, set 
        $f_u(x) \leftarrow \min_z \bigl[ f_{\pi_u}(z) - h_u(z) + \Delta_u(z,x) \bigr] + g_u(x)$
        and set $n \leftarrow n + 1$.
    Then, set
        $F_u(x) \leftarrow F_u(x) + s \times \frac{f_u(x) - F_u(x)}{s_u}$.

\item[S11.] [Update average migration rate.]
    Set $\sigma_P \leftarrow \sigma_P + s \times \frac{\sigma/n - \sigma_P}{s_P}$.

\item[S12.] [Tree loop tail.]
    Go to S2.

\end{description}

We begin in S1 by initializing the cost functions for each node to zero except in
the case of sample nodes, for which we set the cost function to positive infinity for all states
not equal to the observed state. We then set the parent of each node to $-1$ 
(signifying the null element) so that the initial state of the tree sequence
is a forest of disconnected nodes. The average cost functions (denoted by the
corresponding capital letters) are also initialized to $0$ as these will be
updated as we iterate over the tree sequence.

The meat of the algorithm occurs in steps S4 and S5 of the edge removal loop
and in steps S7 and S8 of the edge insertion loop. Removal of an edge from
node $u$ to $v$ will alter the node and stem cost functions along the path from
$u$ back to the root of the genealogy. Step S4 prepares for this by first
subtracting the stem costs of $u$ and $v$ from the node costs of their
respective parents. In step S5, we walk back along the path from $u$'s parent to
the root and recompute new stem costs and node costs given the updated node
costs at the head of the path. Insertion of an edge from node $u$ to $v$ will 
similarly alter the node and stem cost functions along the path from
$u$ back to the root of the genealogy. Step S7 prepares for this by first
substracting the stem cost of $u$ from the node cost of its parent and then
computing the stem cost of $v$ and adding it to the node cost of $u$. In step 
S8, we walk back along the path from $u$'s parent to the root as before and 
recompute new stem costs and node costs given the updated node costs at the 
head of the path.

Upon reaching step S9, we have finished constructing the tree together with its
node cost and stem cost functions. We record the genomic span of the tree in the 
variable $s$, which will be the weight applied to the current tree in the weighted
average. In S10, we perform a preorder traversal of the newly constructed tree
and compute the final cost function for each node. At the same time, we 
increment the total weight $s_u$ of each node by $s$ and update the weighted 
averages $F_u(x)$ with the cost functions for the current tree. When we begin
the traversal at the root(s) of the tree we also record the minimum migration 
cost $\sigma$ needed to explain the sample distribution. We use this cost in
S11, together with the number $n$ of edges in the tree, to update the weighted
average migration rate statistic $\sigma_P$.

Correctness of the algorithm requires that edges are removed in order of 
nondecreasing right genomic coordinate and decreasing time and that edges are
inserted in order of nondecreasing left genomic coordinate and increasing time
(time is measured backward from the present). We assume that the index vectors
$I$ and $O$ are constructed to satisfy these conditions; these assumptions 
form the basis of the conditional checks in steps S3 and S6 for determining
when to exit the edge removal and insertion loops.
