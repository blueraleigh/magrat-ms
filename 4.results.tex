\subsection{Simulations}

Across all simulations, roughly 7 percent of marginal genealogies failed to
coalesce completely resulting in some marginal genealogies having two or (rarely)
three roots. Because our method works with multi-rooted genealogies we did
not recapitate these gene trees.

For simulations generated under a gaussian dispersal kernel, the migration
rate statistic $\sigma_P$ showed a moderate tendency to underestimate the 
effective migration rate, with the magnitude of downward bias increasing with 
with increasing effective migration rate 
(Fig. \ref{fig:gauss-kernel-rate-estimates}). For simulations generated under
a double exponential kernel, the migration rate statistic overestimated the
effective migration rate, with the magnitude of upward bias decreasing with 
with increasing resolution of the discretization
(Fig. \ref{fig:lapl-kernel-rate-estimates}).

Each ancestor $u$'s location was estimated as $\arg\min_x F_u(x)$ and these
estimates are shown in figures \ref{fig:gauss-kernel-ancestor-estimates} and 
\ref{fig:lapl-kernel-ancestor-estimates}. Both the error and the error variance
in location estimates increases for older ancestors, and the rate of increase
with age is faster when migration through the landscape is high. 

\subsection{Empirical application}

We applied our method to a contemporary sample of 1070 georeferenced human
genomes from the Human Genome Diversity Project using a dated tree sequence
inferred for chromosome 20 by \citet{Wohns_etal_2022}. We created an equal 
area discrete global grid \citep{Barnes_Sahr_2023} (cell spacing approximately 
500 km) and allowed for migration between nearest neighbor grid cells. We
chose not to weight the migration cost function by inverse branch length and
assigned each ancestor $u$ a distribution over grid cells by normalizing the 
parsimony weights $w_u(x) = \exp\bigl(-F_u(x) + \min_z F_u(z)\bigr)$.
The analysis finds that the geographic regions with the oldest average ancestor
age are concentrated in northeast Africa and in the Middle East, recapitulating
the likely history of human migration out of Africa and across Asia and Europe 
(Fig. \ref{fig:avg-ancestor-age}). These results broadly agree with 
\citet{Wohns_etal_2022} analysis of the same dataset. By using a discrete global 
grid of Earth's landmass, however, we are able to better incorporate geographic
constraints on dispersal and avoid reconstructing ancestors in places they
clearly did not live (e.g., the Indian Ocean).
