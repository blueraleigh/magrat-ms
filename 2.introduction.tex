The ancestry of a sample of recombining genomes can be described by a complex
network recording the history and timing of coalescent and recombination
events. This network -- the ancestral recombination graph (ARG) -- can be
factored into a sequence of genealogical trees that record the coalescent 
history of the sample for different genomic regions, within which no 
genealogically-determinative recombination has occurred. The tree sequence, 
while not in practice a lossless representation of the ARG, contains a wealth 
of population genetic information about the sample \citep{Ralph_etal_2020}, and
recent advancements in the inference and storage of tree sequences have made 
them a powerful tool for learning about population processes and historical 
events shaping patterns of genetic variation
\citep{Anderson-Trocme_etal_2023, Kelleher_etal_2016, Kelleher_etal_2019}.

The gene trees that comprise a tree sequence are embedded within a pedigree
formed by the dispersal and reproduction of individuals in a landscape. If we 
knew the pedigree and locations of ancestral individuals it would be a simple 
matter to determine how movement patterns have varied over space and time. 
Unfortunately, the pedigree and its geographic context are, in most cases, 
unknown to us. However, the pedigree is intrinsically shaped by the geography 
of individuals \citep{Bradburd_Ralph_2019},
so from the geographic distribution of sampled genomes on the tree
sequence, we can hope to learn about the locations and movements of 
genetic ancestors at different times in the past. The intuition is that most
dispersal and reproduction is geographically constrained: nearby individuals 
tend to be more closely related than far away individuals \citep{Wright_1943}.  
The spatial locations of sampled genomes, together with the genealogies that 
relate them, are therefore expected to be informative about the locations of 
ancestral genomes, at least in the recent past \citep{Wakeley_1999, Wilkins_2004}.

For a non-recombining sequence, simple statistics on the minimum number of
migration events consistent with the gene tree and sample locations have
proven useful for estimating migration rates and other important population
genetic quantities, such as isolation by distance and Wright's neighborhood
size \citep{Slatkin_Maddison_1989, Slatkin_Maddison_1990}. For a recombining
sequence this approach is complicated by the fact that the gene tree on which
the minimum migration statistic is computed necessarily represents some sort of
average of the gene trees of the nonrecombined segments. The work of 
\citet{Hudson_etal_1992} has shown that migration rate estimates based on
minimum migration statistics are surprisingly robust in the presence of
recombination -- often outperforming estimates based on $F_{ST}$ -- but 
suggested a better approach would be to average minimum migration 
statistics across gene trees estimated for each nonrecombined segment.

Our main contribution in this paper is to show how we can use recent
advancements in the storage of tree sequences to efficiently compute and
average minimum migration statistics over genome-wide genealogies. Specifically,
we demonstrate how the generalized parsimony algorithm
\citep{Sankoff_1975, Sankoff_Rousseau_1975} can be used with the succinct tree 
sequence encoding \citep{Kelleher_etal_2016} to estimate migration rates and 
locations of genetic ancestors in continuous and discrete geographic space.
We test this method using spatially explicit forward-time simulations. We then 
apply this method to a continental dataset of humans. Our work adds to a
growing body of research using tree sequences for inference in spatial 
population genomics.





